\documentclass[a4paper,12pt]{article} % тип документа

% Русский язык
\usepackage{wrapfig}
\usepackage[T2A]{fontenc}			
\usepackage[utf8]{inputenc}			
\usepackage[english,russian]{babel}	

\usepackage{indentfirst}
\usepackage[a4paper,top=1.3cm,bottom=2cm,left=1.5cm,right=1.5cm,marginparwidth=0.75cm]{geometry}
\usepackage[usenames]{color}
\usepackage{colortbl}
\usepackage{float}

\usepackage{graphicx}
\usepackage{textcomp}
\usepackage{wrapfig}
\usepackage{hyperref}
\usepackage[rgb]{xcolor}
\usepackage{todonotes}

\usepackage{amsmath,amsfonts,amssymb,amsthm,mathtools} 
\usepackage{wasysym}
\usepackage{euscript}
\usepackage{mathrsfs}

\title{Лабораторная работа №2. Исследование особенностей чисел с плавающей точкой}
\author{Кузин Максим Витальевич}
\date{\today}

\begin{document}
\maketitle
\tableofcontents

В данной работе исследуются способы представления числовых типов в языке программирования C++, а именно беззнаковых целых чисел и чисел с плавающей точкой. Особое внимание уделяется ограничениям формата хранения, возникающим ошибкам округления и нетривиальным эффектам, проявляющимся при вычислениях.

Целью лабораторной работы является анализ внутреннего устройства чисел с плавающей точкой, выявление границ их корректного применения и демонстрация ситуаций, в которых вычисления могут приводить к неожиданным результатам.

\section{Двоичное представление unsigned int}
Для начала рассмотрим, как в памяти хранится беззнаковое целое число. С помощью отладочного вывода отобразим его битовую структуру.

\begin{figure}[H]
    \centering
    \includegraphics[width=0.99\textwidth]{0.png}
\end{figure}

Полученный результат демонстрирует, что значение unsigned int представлено в виде фиксированного набора бит, напрямую соответствующих двоичному представлению числа.

\section{Двоичное представление float}
Аналогичным образом выведем на экран внутреннюю структуру числа с плавающей точкой типа float.

\begin{figure}[H]
    \centering
    \includegraphics[width=0.99\textwidth]{1.png}
\end{figure}

Рассмотрим полученную запись подробнее. Первый бит равен нулю, что указывает на положительный знак числа. Далее следуют 8 бит экспоненты, имеющие значение $10000001$. Оставшиеся биты образуют мантиссу: $01100000000000000000000$.

В результате число представляется в нормализованном виде $101.1$. При переводе в десятичную систему целая часть действительно равна $5$, а дробная часть — $0.5$, что подтверждает корректность интерпретации формата IEEE 754.

\section{Ограничения мантиссы}
Для исследования точности типа float был написан код, последовательно сохраняющий числа вида $10^n$ при увеличении показателя степени $n$.

\begin{figure}[H]
    \centering
    \includegraphics[width=0.99\textwidth]{2.png}
\end{figure}

Результаты выполнения программы показывают, что начиная с некоторого значения $n$ (примерно с $n = 11$), сохранённое число перестаёт точно соответствовать степени десяти. Хотя полученные значения остаются близкими к ожидаемым, погрешность увеличивается с ростом степени.

Это объясняется конечной разрядностью мантиссы и невозможностью точного представления больших десятичных чисел в формате float.

\section{Неожиданное поведение цикла}
Следующий эксперимент демонстрирует влияние ограниченной точности на работу циклов.

\begin{figure}[H]
    \centering
    \includegraphics[width=0.8\textwidth]{3.png}
\end{figure}

В ходе выполнения программы обнаруживается, что при достижении значения $16777216$ переменная цикла типа float перестаёт увеличиваться. В результате условие выхода из цикла никогда не выполняется, и программа зацикливается.

Данный эффект связан с тем, что начиная с этого значения, шаг изменения переменной становится меньше минимально различимого при данной точности числа.

\section{Численное интегрирование}
В рамках работы также было проведено численное интегрирование функции с известным аналитическим результатом. Интеграл вычислялся методом прямоугольников при различном числе разбиений.

\begin{figure}[H]
    \centering
    \includegraphics[width=0.99\textwidth]{04_1.png}
\end{figure}

\begin{figure}[H]
    \centering
    \includegraphics[width=0.99\textwidth]{04_2.png}
\end{figure}

\begin{figure}[H]
    \centering
    \includegraphics[width=0.99\textwidth]{04_3.png}
\end{figure}

Аналитическое значение интеграла равно:
\[
\int_0^1 x^2 \, dx = 0.3333333333333333
\]

Результаты численных экспериментов:

\textbf{FLOAT:}
\begin{itemize}
    \item Оптимальное число разбиений: $N = 5120$
    \item Полученное значение: $0.333235830068588$
    \item Абсолютная ошибка: $9.75 \cdot 10^{-5}$
\end{itemize}

\textbf{DOUBLE:}
\begin{itemize}
    \item Оптимальное число разбиений: $N = 5120$
    \item Полученное значение: $0.333235683441162$
    \item Абсолютная ошибка: $9.76 \cdot 10^{-5}$
\end{itemize}

\section{Заключение}
В ходе выполнения лабораторной работы были изучены особенности представления чисел с плавающей точкой в памяти компьютера. Были выявлены ограничения точности формата float, продемонстрированы примеры накопления ошибок и показано, как они могут приводить к некорректному поведению программ.

Дополнительно были проведены эксперименты по численному интегрированию, наглядно иллюстрирующие влияние разрядности числового типа на точность вычислений.

\end{document}
