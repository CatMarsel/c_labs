\documentclass[a4paper,12pt]{article} % тип документа

% Русский язык
\usepackage{wrapfig}
\usepackage[T2A]{fontenc}            % кодировка
\usepackage[utf8]{inputenc}         % кодировка исходного текста
\usepackage[english,russian]{babel} % локализация и переносы

\usepackage{indentfirst} % красная строка
\usepackage[a4paper,top=1.3cm,bottom=2cm,left=1.5cm,right=1.5cm,marginparwidth=0.75cm]{geometry}
\usepackage[usenames]{color}
\usepackage{colortbl}
\usepackage{float}

\usepackage{graphicx} % картинки
\usepackage{textcomp}
\usepackage{wrapfig}

% гиперссылки
\usepackage{hyperref}
\usepackage[rgb]{xcolor}

% заметки
\usepackage{todonotes}

% математика
\usepackage{amsmath,amsfonts,amssymb,amsthm,mathtools}
\usepackage{wasysym}
\usepackage{euscript}
\usepackage{mathrsfs}

\title{Лабораторная работа 2. Особенности работы чисел с плавующей точкой}
\author{Кузин Максим Витальевич}
\date{\today}

\begin{document}
\maketitle

\tableofcontents

\section{Введение}

В рамках данной лабораторной работы проводится исследование различных алгоритмов сортировки с точки зрения их вычислительной эффективности и асимптотической сложности. Рассматривается влияние объёма входных данных, а также их начальной упорядоченности, на время выполнения алгоритмов. В ходе работы анализируются как простые методы сортировки, так и более производительные алгоритмы, применяемые на практике, а именно: пузырьковая сортировка, сортировка выбором, сортировка вставками, быстрая сортировка, сортировка кучей и сортировка слиянием.

Основной задачей лабораторной работы является закрепление теоретических знаний об асимптотической сложности алгоритмов, а также получение практического опыта в их программной реализации и экспериментальном сравнении. Проведённый анализ позволяет наглядно показать, насколько критичен правильный выбор алгоритма при обработке больших массивов данных в реальных вычислительных задачах.

\subsection{Эксперимент}

Экспериментальные измерения выполнялись на персональном компьютере, оснащённом процессором Intel i5. Температурный режим поддерживался постоянным.
\section{Алгоритмы с асимптотикой $O(n^2)$}

\subsection{Демонстрация}

В первую очередь рассмотрим алгоритмы сортировки, отличающиеся простотой реализации, но обладающие квадратичной временной сложностью.
\begin{figure}[H]
    \centering
    \includegraphics[width=0.7\textwidth]{img/first/O2.jpg}
    \caption{Асимптотическая сложность $O(n^2)$}
\end{figure}

Для подтверждения квадратичного характера роста времени выполнения используется логарифмическое представление данных:
\[
\ln(t) = \ln(C) + 2\ln N.
\]

\begin{figure}[!h]
    \centering
    \includegraphics[width=0.7\textwidth]{img/first/O2ln.jpg}
    \caption{Подтверждение сложности $O(n^2)$ в логарифмическом масштабе}
\end{figure}

Графики в логарифмическом масштабе имеют линейный вид, что подтверждает квадратичную зависимость времени выполнения от размера входных данных.

\subsection{Сравнение оптимизаций}

\begin{figure}[H]
    \centering
    \includegraphics[width=0.7\textwidth]{img/first/optimO2.jpg}
    \caption{Влияние оптимизаций на алгоритмы с квадратичной сложностью}
\end{figure}

Даже при применении оптимизаций алгоритмы данного класса остаются малоэффективными при увеличении размера массива.

\section{Алгоритмы с асимптотикой $O(N\log N)$}

\subsection{Демонстрация}

Далее рассмотрим более эффективные алгоритмы сортировки и сравним их между собой.
\begin{figure}[H]
    \centering
    \includegraphics[width=0.55\textwidth]{img/first/qw.jpg}
    \caption{Асимптотическая сложность $O(N\log N)$}
\end{figure}

Для проверки логарифмической зависимости используется следующая нормализация:
\[
t = \frac{t}{N \ln N}.
\]

\begin{figure}[H]
    \centering
    \includegraphics[width=0.7\textwidth]{img/first/qvln.jpg}
    \caption{Подтверждение сложности $O(N\log N)$}
\end{figure}

Полученные графики имеют близкий к линейному характер, что подтверждает соответствие экспериментальных данных теоретической оценке сложности.

\subsection{Сравнение оптимизаций}

\begin{figure}[!h]
    \centering
    \includegraphics[width=0.5\textwidth]{img/first/qwoptim.jpg}
    \caption{Эффект оптимизаций для алгоритмов $O(N\log N)$}
\end{figure}

Быстрая сортировка демонстрирует наилучшие показатели производительности: даже без оптимизаций она опережает остальные алгоритмы данного класса.

\section{Сравнение $O(N\log N)$ и $O(N^2)$}

Для наглядного сопоставления алгоритмов различных классов сложности были отключены все оптимизации.
\begin{figure}[H]
    \centering
    \includegraphics[width=0.7\textwidth]{img/first/all.jpg}
    \caption{Сравнение всех исследуемых алгоритмов сортировки}
\end{figure}

Различие в скорости работы алгоритмов становится особенно заметным при росте размера входных данных.

\section{Зависимость от начального состояния данных}

Для анализа влияния структуры входных данных использовались три типа массивов: случайно упорядоченные, отсортированные по возрастанию и отсортированные по убыванию.
\begin{figure}[H]
    \centering
    \includegraphics[width=0.99\textwidth]{img/tr1.jpg}
\end{figure}

\begin{figure}[H]
    \centering
    \includegraphics[width=0.99\textwidth]{img/tr2.jpg}
\end{figure}

Результаты показывают, что начальное расположение элементов может существенно влиять на время работы отдельных алгоритмов.

\section{Вывод}

В процессе выполнения лабораторной работы была проведена экспериментальная и теоретическая оценка эффективности различных алгоритмов сортировки, включая пузырьковую сортировку, сортировку выбором, сортировку вставками, быструю сортировку, сортировку кучей и сортировку слиянием.

Было исследовано время выполнения алгоритмов на массивах различной структуры: случайных, упорядоченных и обратных. Эксперимент подтвердил, что временная сложность существенно зависит как от размера данных, так и от их начального состояния.

Алгоритмы с квадратичной сложностью $O(n^2)$, такие как пузырьковая сортировка и сортировка выбором, показали крайне низкую эффективность при работе с большими массивами и практически непригодны для реальных задач.

Алгоритмы с асимптотикой $O(N\log N)$ продемонстрировали значительно более высокую производительность. Среди них быстрая сортировка оказалась наиболее эффективной во всех проведённых тестах.

Полученные результаты наглядно показывают важность анализа алгоритмической сложности и обоснованного выбора алгоритмов при разработке программного обеспечения и оптимизации вычислительных процессов.
\end{document}
